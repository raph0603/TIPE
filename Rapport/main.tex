\documentclass[12pt,a4paper]{article}
%----------------------------------------------------------------
% DÉBUT DE L'ENTÊTE
% À IGNORER EN PREMIÈRE LECTURE
%----------------------------------------------------------------
\usepackage{url}            % Pour citer les adresses web
\usepackage[T1]{fontenc}    % Encodage des accents
\usepackage[utf8]{inputenc} % Lui aussi
\usepackage[french]{babel} % Pour la traduction française
\usepackage{amsmath}        % La base pour les maths
\usepackage{mathrsfs}       % Quelques symboles supplémentaires
\usepackage{amssymb}        % encore des symboles.
\usepackage{amsfonts}       % Des fontes, eg pour \mathbb.

\usepackage[svgnames]{xcolor} % De la couleur
\usepackage{geometry}       % Gérer correctement la taille



\newcounter{nextyear}
\setcounter{nextyear}{\the\year}
\addtocounter{nextyear}{1}

% Mettez votre titre de TIPE et votre nom ci-après
\title{Prévisions : l'Enjeux d'un Futur Durable (à voir avec le collègue)}
\author{Raphaël Laborie, MP\oldstylenums{1}-MPi, \oldstylenums{\the\year}-\oldstylenums{\arabic{nextyear}} }
%% À décommenter si vous ne voulez pas que la date apparaisse explicitement
%\date{}

% Un raccourci pour composer les unités correctement (en droit)
% Exemple: $v = 10\U{m.s^{-1}}$
\newcommand{\U}[1]{~\mathrm{#1}}

% Pour discuter avec le prof dans le document: le premier argument est 
% le nom de celui qui fait la remarque et le second la remarque 
% proprement dite: \question{jj}{Que voulez-vous dire par là ?}
% \reponse{Droopy}{I'm very happy...}
\usepackage{todonotes}
\newcommand{\question}[2]{\todo[inline,author=#1]{#2}}
\newcommand{\reponse}[2]{\todo[inline,color=green,author=#1]{#2}}

% Les guillemets \ofg{par exemple}
\newcommand{\ofg}[1]{\og{}#1\fg{}}
% Le d des dérivées doit être droit: \frac{\dd x}{\dd t}
\newcommand{\dd}{\text{d}}



% NB: le script TeXcount permet de compter les mots utilisés dans chaque section d'un document LaTeX. Vous en trouverez une version en ligne à l'adresse
% http://app.uio.no/ifi/texcount/online.php
% Il suffit d'y copier l'ensemble du présent document (via Ctrl-A/Ctrl-C puis Ctrl-V dans la fenêtre idoine) pour obtenir le récapitulatif tout en bas de la page qui s'ouvre alors.

% Pour récupérer les bonnes entrées bibliographiques, je vous conseille l'usage de scholar.google.fr pour les recherches
% et la récupération des entrée BibTeX comme décrit dans cette vidéo: https://www.youtube.com/watch?v=X-9T2Oaj-5A

\newcommand{\positionnementThematique}[1]{
\section*{Positionnement thématique}
{\it #1}}

\newcommand{\motclefs}[2]{
    \section*{Mots-clefs}
        \begin{description}
            \item[Mots-clefs] -- #1 
            \item[Keywords]   -- #2
        \end{description}
}
\setlength {\marginparwidth }{2cm}
%-----------------------------------------------------------------------------
% FIN DE L'ENTÊTE
%-----------------------------------------------------------------------------
\begin{document}

\maketitle
%--------------------------------------------------
\section*{Motivations pour le choix du sujet}
%--------------------------------------------------
Les énergies renouvelables sont un défi de plus en plus important dans un monde en constante transformation et dont les ressources deviennent limitées.
Les smart grids offrent une nouvelle approche sur la gestion de l'énergie et posent plus que jamais la question d'une anticipation fine des consommations électriques.
%--------------------------------------------------
\section*{Ancrage du sujet au thème de l'année}
%--------------------------------------------------
Les villes, principales places du développement des smart grids, voient naître plusieurs de ces nouveaux systèmes de gestion d'énergie. Donnant lieu à un problème d'optimisation, l'anticipation de la consommation électrique des villes constitue le défi majeur dans le fonctionnement de beaucoup de smart grids.
%--------------------------------------------------
\positionnementThematique{Informatique théorique, Informatique pratique, Mathématiques appliquées}
%--------------------------------------------------
\motclefs{Prévisions -- Graphes de Visibilités -- Méthodes Régressives -- Moyenne Mobile Auto-Régressive Intégrée (ARIMA) -- Réseaux Intelligents}{Forecasting -- Visibility Graphs -- Regressive Methods -- Auto-Regressive Integrated Moving Average (ARIMA) -- Smart Grids}
%--------------------------------------------------
\section*{Bibliographie commentée (650 mots maximum)}
%--------------------------------------------------
% D'après TeXcount, la section fait 366 mots: 
% 366+3+0 (1/0/4/0)

La prédiction est un domaine qui s'est étendu durant les dernières années et qui a profité aussi bien de l'évolution des techniques que de la recherche dans ce domaine. Si dernièrement les méthodes les plus utilisées sont basées sur l'utilisation de Réseaux de Neurones Artificiels, ou Artificial Neural Newtork (ANN) en anglais, certains modèles plus anciens (mais qui restent malgré tout parmis les plus utilisés) sont les régressions telles que les régressions polynomiales \cite{polyregrEva} (Polynomial Regression), les moyennes mobiles (Moving Average) et, les régressions exponentielles (Exponential Smoothing), une forme plus avancée de moyenne mobile \cite{exporeview}.

En parallèle se développe la methode ARIMA (Auto Regressive Integrated Moving Average, ou Moyenne Mobile Intégrée Auto Régressive). Après la publication du livre de George Box et Gwylim Jenkins en 1970 \cite{boxjenkins}, elle est largement démocratisée et voit naître avec elle une méthode, aujourd'hui encore relativement répendue. Cette méthode permet de vérifier la précision d'un modèle. Elle est appelée "Box-Jenkins" en hommage à ses créateurs.

En terme de prévisions, la méthode de régression linéaire est un modèle assez simple à mettre en place, mais qui apporte bien souvent des résultats non-fidèles à la réalité. Ainsi, la méthode de régression polynomiale, une généralisation aux polynômes de degrés quelconques, apporte une meilleure précision. Elle est particulièrement efficace pour des prévisions très localisées. Cette méthode s'appuye sur la méthode des moindre carrés \cite{polyregrEva,polyregrElias} décomposant le problème en plusieurs problèmes de régression linéaire. 

Les applications de ces méthodes sont diverses et variées. Leurs domaines d'application vont de l'anticipation du cours de la monnaie \cite{zambiaES} à la prévision du comportement de la production de riz en Indonésie.

L'économie actuelle est sujette à de nombreuses fluctuation, ce qui en fait un sujet propice à l'application de la méthode ARIMA. La recherche d'un modèle adapté \cite{hughchristensen} à cette application \cite{economyARIMA} est une étape importante lors de la mise en place de l'algorithme. Très utile pour une application sur les séries temporelles (time series), elle s'appuye principalement sur l'autocorrélation des données et les motifs récurrents.

Une autre approche pour la prévision est la méthode Exponential Smoothing \cite{exporeview}. Beaucoup utilisée pour des applications économiques, elle convient très bien à des prévisions sur le court terme. Contrairement à la méthode ARIMA, cette méthode tient compte des comportements saisonniers et des tendances et convient en particulier pour des données stationnaires.

Il existe, également des méthodes plus récentes, comme les graphes de visibilité \cite{visigraphPhyA, visigraphTEPJB}. Cette méthode utilise le principe de similarité, en transformant un jeu de données/séries temporelles (time series) en un graphe \cite{visigraphPNAS} dont les arêtes représentent les similarités entre les données. Comme pour la méthode ARIMA, cette méthode est basée sur la reconnaissance de motifs. Elle est particulièrement efficace pour des prévisions sur le long terme. 

%--------------------------------------------------
\section*{Problématique retenue (50 mots)}
%--------------------------------------------------
 Les smart grids sont de plus en plus utilisées pour contrôller la distribution de l'électricité dans les villes et elles nécessitent toutes d'anticiper la consommation en électricité pour adapter la production à la demande. Plusieurs méthodes sont proposées pour anticiper cette consommation d'énergie. On se demandera alors, au moyen d'une étude comparative : avec quelles méthodes et dans quelles conditions est-il possible d'anticiper la consommation électrique des villes ?
%--------------------------------------------------
\section*{Objectifs du TIPE (100 mots maximum)}
%--------------------------------------------------
\begin{enumerate}
    \item   Compréhension du problème et de ses enjeux pour choisir un modèle adapté.

    \item   Etude de la méthode de régression polynomiale et application à des données simples.
	
    \item   Application de la méthode des Graphes de Visibilité : par la construction d'un graphe de visibilité, la définition et l'identification des similarités puis, l'annonce des prévisions sur des données simples.
    
    \item	Analyse des résultats obtenus : donner les domaines de fonctionnement optimales de chaque méthode en fonction des paramètres apportés. Rendre compte alors des méthodes pour déterminer la plus efficaces.
    
    \item   Confrontation des algorithmes à des données réelles, et études des limites et améliorations possibles de chaque méthode.
\end{enumerate}
%--------------------------------------------------
%--------------------------------------------------
% Pour faire apparaître la bibliographie avec des chiffres, 
% dans l'ordre d'apparition dans le texte
\bibliographystyle{unsrt-fr}     % Style de la bibliographie (numérotée dans l'ordre d'apparition du texte)
\bibliography{biblio} % Nom du fichier .bib à utiliser


\end{document}