\documentclass[12pt,a4paper]{article}
%----------------------------------------------------------------
% DÉBUT DE L'ENTÊTE
% À IGNORER EN PREMIÈRE LECTURE
%----------------------------------------------------------------
\usepackage{url}            % Pour citer les adresses web
\usepackage[T1]{fontenc}    % Encodage des accents
\usepackage[utf8]{inputenc} % Lui aussi
\usepackage[french]{babel} % Pour la traduction française
\usepackage{amsmath}        % La base pour les maths
\usepackage{mathrsfs}       % Quelques symboles supplémentaires
\usepackage{amssymb}        % encore des symboles.
\usepackage{amsfonts}       % Des fontes, eg pour \mathbb.

\usepackage[svgnames]{xcolor} % De la couleur
\usepackage{geometry}       % Gérer correctement la taille



\newcounter{nextyear}
\setcounter{nextyear}{\the\year}
\addtocounter{nextyear}{1}

% Mettez votre titre de TIPE et votre nom ci-après
\title{Titre de mon TIPE}
\author{Raphaël Laborie, MP\oldstylenums{1}-MPi, \oldstylenums{\the\year}-\oldstylenums{\arabic{nextyear}} }
%% À décommenter si vous ne voulez pas que la date apparaisse explicitement
%\date{}

% Un raccourci pour composer les unités correctement (en droit)
% Exemple: $v = 10\U{m.s^{-1}}$
\newcommand{\U}[1]{~\mathrm{#1}}

% Pour discuter avec le prof dans le document: le premier argument est 
% le nom de celui qui fait la remarque et le second la remarque 
% proprement dite: \question{jj}{Que voulez-vous dire par là ?}
% \reponse{Droopy}{I'm very happy...}
\usepackage{todonotes}
\newcommand{\question}[2]{\todo[inline,author=#1]{#2}}
\newcommand{\reponse}[2]{\todo[inline,color=green,author=#1]{#2}}

% Les guillemets \ofg{par exemple}
\newcommand{\ofg}[1]{\og{}#1\fg{}}
% Le d des dérivées doit être droit: \frac{\dd x}{\dd t}
\newcommand{\dd}{\text{d}}



% NB: le script TeXcount permet de compter les mots utilisés dans chaque section d'un document LaTeX. Vous en trouverez une version en ligne à l'adresse
% http://app.uio.no/ifi/texcount/online.php
% Il suffit d'y copier l'ensemble du présent document (via Ctrl-A/Ctrl-C puis Ctrl-V dans la fenêtre idoine) pour obtenir le récapitulatif tout en bas de la page qui s'ouvre alors.

% Pour récupérer les bonnes entrées bibliographiques, je vous conseille l'usage de scholar.google.fr pour les recherches
% et la récupération des entrée BibTeX comme décrit dans cette vidéo: https://www.youtube.com/watch?v=X-9T2Oaj-5A

\newcommand{\positionnementThematique}[1]{
\section*{Positionnement thématique}
{\it #1}}

\newcommand{\motclefs}[2]{
    \section*{Mots-clefs}
        \begin{description}
            \item[Mots-clefs] -- #1 
            \item[Keywords]   -- #2
        \end{description}
}
\setlength {\marginparwidth }{2cm}
%-----------------------------------------------------------------------------
% FIN DE L'ENTÊTE
%-----------------------------------------------------------------------------
\begin{document}

\maketitle
%--------------------------------------------------
\section*{Motivations pour le choix du sujet}
%--------------------------------------------------
Les énergies renouvelables sont un défi de plus en plus important dans un monde en constante transformation et dont les ressources deviennent limitées.
Les smart grids offrent une nouvelle approche sur la gestion de l'énergie et posent plus que jamais la question d'une anticipation fine des consommations électriques.
%--------------------------------------------------
\section*{Ancrage du sujet au thème de l'année}
%--------------------------------------------------
Les villes, principales places du développement des smart grids, voient naître plusieurs de ces nouveaux systèmes de gestion d'énergie. Donnant lieu à un problème d'optimisation, l'anticipation de la consommation électrique des villes constitue le défi majeur dans le fonctionnement de beaucoup de smart grids.
%--------------------------------------------------
\positionnementThematique{Informatique théorique, Informatique pratique, Mathématiques appliquées}
%--------------------------------------------------
\motclefs{Prévisions -- Réseaux intelligents -- Graphes de Visibilités -- Régression polynomiale}{Forecasting -- Smart grids -- Visibility Graphs -- Polynomiale Regression}
%--------------------------------------------------
\section*{Bibliographie commentée (650 mots maximum)}
%--------------------------------------------------
% D'après TeXcount, la section fait 366 mots: 
% 366+3+0 (1/0/4/0)

L'économie actuelle est sujette à de nombreuses fluctuation, ce qui en fait un sujet propice à l'application de la méthode connue sous le nom de ARIMA. La recherche d'un modèle adapté \cite{hughchristensen} à l'application de cette meme méthode \cite{economyarima} est une étape importante lors de la mise en place de l'algorithme. Très adaptée à une application sur les séries temporelles, cette méthode s'appuye beaucoup sur l'autocorrélation des données et les patternes récurrents.

Une autre approche pour la prévision est la méthode Exponential Smoothing \cite{exporeview} \cite{exporeview2} . Beaucoup utilisée pour des applications économiques \cite{zambiaeconomy}, elle convient très bien à des prévisions sur le court terme. Contrairement à la méthode ARIMA, cette méthode tient compte des comportements saisonniers et des tendances.

%--------------------------------------------------
\section*{Problématique retenue (50 mots)}
%--------------------------------------------------
 Les smart grids sont de plus en plus utilisées pour contrôller la distribution de l'électricité dans les villes et elles nécessitent toutes d'anticiper la consommation en électricité pour adapter la production à la demande. Plusieurs méthodes sont proposées pour anticiper cette consommation d'énergie. On se demandera alors, au moyen d'une étude comparative : avec quelles méthodes et dans quelles conditions est-il possible d'anticiper la consommation électrique des villes ?
%--------------------------------------------------
\section*{Objectifs du TIPE (100 mots maximum)}
%--------------------------------------------------
\begin{enumerate}
    \item   Identification du problème: Une compréhension claire du problème permet de chosir un modèle adapté à la prévision de la consommation et au traitement des données.

    \item   Elaboration d'un modèle simple: Un modèle simple de régression polynomiale donnera une référence basique pour comparer les résultats obtenus après application des algorithmes traités lors du projet.
	
    \item   Traitement des données et algorithmique: Une fois le problème scerné et le modèle choisi, le but sera d'appliquer les méthodes ARIMA et Exponential Smoothing pour obtenir une prévision de la consommation sur court et long terme.
    
    \item	Comparaison des résultats: Après la comparaison des résultats des différents algorithmes entre eux et avec le modèle simple, le but sera d'identifier les points forts et les points faibles de chaque algorithme et d'interpréter les résultats, en donnant une idée de l'efficacité de chacun d'entre eux.
    
    \item   Amélioration des algorithmes: L'identification précédente des points forts et faibles des algorithmes et leur mutuelle complétion permettra de proposer une approche combinant la méthode ARIMA et la méthode Exponential Smoothing.
\end{enumerate}
%--------------------------------------------------
%--------------------------------------------------
% Pour faire apparaître la bibliographie avec des chiffres, 
% dans l'ordre d'apparition dans le texte
\bibliographystyle{unsrt-fr}     % Style de la bibliographie (numérotée dans l'ordre d'apparition du texte)
\bibliography{biblio} % Nom du fichier .bib à utiliser


\end{document}